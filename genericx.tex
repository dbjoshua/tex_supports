\ProvidesFile{lyx-generic.cfg}

% todo ling-packages

\RequirePackage{qtree,stmaryrd}
\RequirePackage{iftex}
  \ifXeTeX
    \RequirePackage{xunicode}
  \else
    \RequirePackage{tipa}
  \fi 

\RequirePackage[normalem]{ulem}
\RequirePackage{xcolor,graphicx}

% todo generic text formatting 
% =======================
% GENERIC TEXT FORMATTING
% =======================

% CAPITALIZATION
% šššššššššššššš

% Please follow the instructions in the Generic Style Rule

% ITALICS
% ššššššš
%:> 'techterm'
\makeatletter%
\newcommand*{\techterm@unstarred}[1]{\textit{#1}}
\newcommand*{\techterm@starred}[1]{``#1''}
\newcommand*{\techterm}{\@ifstar{\techterm@starred}{\techterm@unstarred}}
\makeatother% 
%<:
\def\olform#1{\textit{#1}}
\let\ol\olform

\def\phonet#1{{[}#1{]}}
\let\pht\phonet
\def\phonol#1{/#1/}
\let\phl\phonol

\def\lsemph#1{\textit{#1}}
\let\lem\lsemph
\let\emph\lsemph\relax
%\newcommand{\emph}[1]{\textit{#1}}% forces \emph to behave like \textit all the time
\let\em\itshape% forces \em to behave like \itshape all the time

\def\titre#1{\textit{#1}}
\def\metalang#1{\textit{#1}}

% SMALL CAPS
% šššššššššš
\def\stress#1{\textsc{\MakeLowercase{#1}}}
\def\kterm#1{\textsc{\MakeLowercase{#1}}} % important term at its first use (cmd to be modified so as to index such terms)
\ifdefined\gram
  \def\gr#1{\gram{\MakeLowercase{#1}}}
  \else \def\gr#1{\textsc{\MakeLowercase{#1}}}
  \fi%

% BOLDFACE & OTHER HIGHLIGHTINGS
% šššššššššššššššššššššššššššššš
%:> 'gshl' % command pour la mise en emphase/relief
\newcommand{\gshl}[2][]{%
\ifthenelse{\equal{#1}{s}}
{\underline{#2}}
{\textbf{#2}}
}
%<:
\let\hl\gshl 
\let\pem\gshl
\let\pemph\pem
\let\highlight\gshl

% QUOTATION MARKS
% ššššššššššššššš

%:> quotrans
\newcounter{mot}
\def\maxquotelen{39}
\def\transpreamble{Nous traduisons:~}

\def\qml{« }
\def\qmr{ »}

\begingroup
\lccode`\~=`\ %
\lowercase{%
\gdef\quotrans{\setcounter{mot}{0}%
\qml\hskip-1em
\catcode`~=\active
\def~{\space\stepcounter{mot}}
}}%
\endgroup%
\def\endquotrans{
\hskip-.7em\qmr
\stepcounter{mot}
\ifnum \value{mot}>\maxquotelen
\PackageError{gl-com-cord-cf}{Citation too long! Use \footqtrans command instead!}{Citation too long! Use \footqtrans command instead!}
\else\fi
}
\begingroup
\lccode`\~=`\ %
\lowercase{%
\gdef\inlinequotation{\setcounter{mot}{0}%
\qml\hskip-1em
\catcode`~=\active
\def~{\space\stepcounter{mot}}
}}%
\endgroup%
\def\endinlinequotation{
\hskip-.7em\qmr
\stepcounter{mot}
\ifnum \value{mot}>\maxquotelen
\PackageError{gl-com-cord-cf}{Citation too long! Use "quote" environment instead!}{Citation too long! Use "quote" environment instead!}
\else\fi
}
\newcommand{\inlinequote}[1]{\inlinequotation~ #1~ \endinlinequotation}
\newenvironment{quotetrans}{\quotrans}{\endquotrans}
\newcommand{\footqtrans}[1]{\footnote{\qml~ #1~ \qmr}}
\newcommand{\qtrans}[1]{\begin{quotetrans}#1\end{quotetans}}
%<:
\def\qell{[$\dots$]}
\def\qemmark{[emphasis mine]} % use it at the end of a quotation if it contains emphases.

% |  Uncomment the following  
% v  two lines for english style quotes
%\renewcommand\qml{``}
%\renewcommand\qmr{''}

% //Linguistic meaning
\def\mean#1{`#1'}
\let\signif\mean

% OTHER PUNCTUATION
% .................

\newcommand{\letter}[1]{{\LARGE\raisebox{-0.2ex}\guilsinglleft}\hskip-.28ex #1\hskip-.28ex{\LARGE\raisebox{-0.2ex}\guilsinglright}} % curly angle brackets defined to distinguish reference to symbols and semantic types for example.

% ABBREVIATIONS
% ššššššššššššš

% ! Under construction !
% ToDo : mecanism to typeset abbreviations
% Option 1 : gloss
% Option 2 : leipzig
% the package should allow having an external database
% which will be called upon while allowing the list of
% abreviations not to be printed out.
% I chose option 1
% \RequirePackage{gloss}
% customize "headings" so as to be able to completely hide them
% remove the dot after the definition in the list of abbreviations

\makeatother

% todo generic numex 
% =======================
% GENERIC NUMEX
% =======================


% todo generic_numex
% ^- ------------- _- 
% ^/ GENERIC NUMEX _/ 
% ^- ------------- _- 

% =======================================================
% = THIS IS FILE	 `generic_numex.tex 				=
% = FOR		formatting numbered examples accordint	 	=
% = 		to the	Generic Style Rules for Linguistics	=
% =======================================================

% ^/ FOR configurations for generic_numex
% Erasing_ex_Command
\makeatletter 
\@ifpackageloaded{covington}{
% Erasing_ex.and.bx_Cmds
\renewcommand{\bx}{}
\renewcommand{\ex}{}
% End
% Defining_New_Cmds
\newcommand{\bgx}{\begin{covexercise}}
\newcommand{\edx}{\end{covexercise}}
% End
}{\relax}
\@ifpackageloaded{linguex}{
% Erasing_ex_Cmd
\renewcommand{\ex}{}
% End
}{\relax}
\@ifpackageloaded{gb4e}{
% Erasing_ex_Cmd
\renewcommand{\ex}{}
% End
}{\relax}
\makeatother 
% End 

\makeatletter
% Defining_Macro_For_Activation_Of_Chars
\def\gUmakeletter#1{\catcode`#1=11\relax} 
%\def\@makeother#1{\catcode`#1=12\relax}
\def\makeusletter{\gUmakeletter{_}}
\def\restoreus{\catcode`\_=8}
\makeusletter
\let\g_makeletter\gUmakeletter\relax
\def\g_xs_activate#1{\catcode`#1=\active\relax}
\let\g_xs_deactivate\g_makeletter\relax
% End
\makeatletter

%
% EXAMPLE
%

% Defining_Setting_Commands
\def\g_ex_num_sep{0pt}

% ------------------------------------------------------% 
%   To let 'lex' know whether it is within a footnote  	%
%   Copied from 'linguex.sty'						    %
%														%
\newif\if@noftnote\@noftnotetrue%						%
\let\predefinedfootnotetext=\@footnotetext%				%
\long\def\@footnotetext#1{%								%
\@noftnotefalse\predefinedfootnotetext{#1}%			%
\@noftnotetrue}%									%
% ------------------------------------------------------%
% DefiningCounters
% Defining_Conditionals_For_Number_Appearence
\newif\iflexnumb\lexnumbfalse% 
% lexno without parenthesis
\newif\ifsublexnumb\sublexnumbfalse% 
% sublexno plus lexno without parentheses -- e.g. 1a
\newif\ifsublexnumo\sublexnumofalse% 
% sublexno only e.g. a
% End
\newcounter{lexno}
\renewcommand\thelexno{(\arabic{lexno})}
%\let\thelexno\g_the_lexno%
\newcounter{fnlexno}
\renewcommand\thefnlexno{(\roman{fnlexno})}
% End
% Defining_lex_Env
\newenvironment{lex}{
\begin{enumerate}
\if@noftnote\refstepcounter{lexno}
\else\refstepcounter{fnlexno}\fi 
\addtolength{\labelsep}{\g_ex_num_sep}%
\setlength{\listparindent}{0pt}%
\def\makelabel##1{##1\hfil}%       % put labels flushleft in space available
\if@noftnote\item[(\arabic{lexno})]
\else\item[(\alph{fnlexno})]
\fi 
}{\end{enumerate}}
% End


%
% SUBEXAMPLE
%

% DefiningCounters
\newcounter{sublex}[lexno]
\renewcommand\thesublex{(\arabic{lexno}\alph{sublex})}
\newcounter{fnsublex}[fnlexno]
\renewcommand\thefnsublex{(\roman{fnlexno}\alph{fnsublex})}
% End
% Defining_subex_Env
\newenvironment{subex}{
\vspace*{-.4em}
\begin{enumerate}
\if@noftnote\refstepcounter{sublex}
\else\refstepcounter{fnsublex}\fi 
\addtolength{\labelsep}{\g_ex_num_sep}%
\setlength{\listparindent}{0pt}%
\def\makelabel##1{##1\hfil}%       % put labels flushleft in space available
\if@noftnote\item[\alph{sublex}.]
\else\item[\alph{fnsublex}.]
\fi 
}{
\end{enumerate}
%\vspace*{-.5em}
}
% End
%:> 'ex'
% Redefining_ex_Cmd
\makeatletter%
\newcommand*{\ex@unstarred}{\begin{lex}}
\newcommand*{\ex@starred}{\end{lex}}
\newcommand*{\ex}{\@ifstar{\ex@starred}{\ex@unstarred}}
\makeatother%
%<:
%:> 'ix'
\makeatletter%
\newcommand*{\ix@unstarred}{\begin{subex}}
\newcommand*{\ix@starred}{\end{subex}}
\newcommand*{\ix}{\@ifstar{\ix@starred}{\ix@unstarred}}
\makeatother%
%<: 

%
% EXAMPLE (with title)
% 
\newenvironment{tlex}[1]{\begin{lex}\exhead{#1}}{\end{lex}}

%
% EXAMPLE (italic font)
%

%:> 'exi'
\makeatletter%
\newcommand*{\exi@unstarred}{\itshape\begin{lex}}
\newcommand*{\exi@starred}{\end{lex}\normalfont}
\newcommand*{\exi}{\@ifstar{\exi@starred}{\exi@unstarred}}
\makeatother%
\newenvironment{lexi}{\exi}{\exi*}
%<:

%
% SUBEXAMPLE (italic font)
%

%:> 'ixi'
\makeatletter%
\newcommand*{\ixi@unstarred}{\itshape\begin{subex}}
\newcommand*{\ixi@starred}{\end{subex}\normalfont}
\newcommand*{\ixi}{\@ifstar{\ixi@starred}{\ixi@unstarred}}
\newenvironment{subexi}{\ixi}{\ixi*}
\makeatother%
%<: 


%
% LOW LEVEL COMMANDS FOR NUMBERED EXAMPLES
%

%:> 'langinfo' : cs for typesetting language informations
\def\langinfofn#1{\textup{#1}}
\newcommand{\langinfo}[2][]{\ifthenelse{\equal{#1}{}}{\langinfofn{#2}}{\langinfofn{#2 $($#1$)$}}}
\newcommand{\jlanginfo}[2][]{
\ifthenelse{\equal{#1}{}}{\jl{#2}\def\endlex{\printlanguage\end{enumerate}}}{\jl{#2, #1}}\def\endlex{\printlanguage\end{enumerate}}
}%<:    
%:> 'exhead'
\makeatletter
\let\boldface\textbf
\let\gras\textbf
\let\italiques\textit
\let\italics\textit
\newcommand{\exheadfn}[1]{\gras{\textup{#1}}}
\newcommand*{\exhead@unstarred}[1]{\exheadfn{#1}\\}
\newcommand*{\exhead@starred}[1]{\exheadfn{#1}}
\newcommand*{\exhead}{\@ifstar{\exhead@starred}{\exhead@unstarred}}
\makeatother 
%<:

%
% LOW LEVEL COMMANDS FOR GLOSSED EXAMPLES
%

% Renaming_Commands_From_cgloss.sty

% ---------------------------------------------------
% - The file cgloss.sty was copied and modified 	-
% - from the Glossalatex template bundle 			-
% - to match instructions in the 					-
% - Generic Style Rule for Linguitics				-
% - By Akpoué Kouamé Josué 							-
% - (professional gmail addr)						-
% ---------------------------------------------------

%%%
%%% Sentences with word-by-word glosses
%%%

\let\g_gl_single=1
\def\singlegloss{\let\g_gl_single=1}
\def\nosinglegloss{\let\g_gl_single=0}
\def\@selfnt{\selectfont}


\def\g_ex_language{}
\newcommand{\jl}[1]{% Language info into jambox
\def\g_ex_language{#1}
}
\def\printlanguage{
\ifthenelse{\equal{\g_ex_language}{}}
{}{\hfill(\g_ex_language)\hspace*{-1cm}}}


\def\g_gl_lines_single%                  % Introduces 2-line text-and-gloss.
{\begingroup\bgroup %\begin{flushleft}
\ifx\g_gl_single1%           conditionally force single spacing (hpk/MC)
\def\baselinestretch{1}\selectfont\fi
%        \vskip\baselineskip\def\baselinestretch{1}%
%        \@selfnt\vskip-\baselineskip\fi%
\bgroup
\g_gl_twosent
}

\def\g_gl_lines_double%                  % Introduces 3-line text-and-gloss.
{\begingroup\bgroup %\begin{flushleft}
\ifx\g_gl_single1%        conditionally force single spacing (hpk/MC)
\def\baselinestretch{1}\selectfont\fi
%        \vskip\baselineskip\def\baselinestretch{1}%
%        \@selfnt\vskip-\baselineskip\fi%
\bgroup
\g_gl_threesent
}

% \def\glt{\vskip.0\baselineskip}

% redefine \gltoffset to set off translation from ex and gloss
\def\gltoffset{0pt}

\def\g_gl_tr{\ifhmode\\*[\gltoffset]\else\nobreak\vskip\gltoffset\nobreak\fi}


% Introduces a translation
\let\trans\g_gl_tr%

\def\gln{\relax}
\def\glend{} % obsolete
% Ends the gloss environment.

\newbox\g_gl_line_one % boxes with words from first line
\newbox\g_gl_line_two
\newbox\g_gl_line_three
\newbox\g_gl_word_one % a word from the first line (hbox)
\newbox\g_gl_word_two
\newbox\g_gl_word_three
\newbox\g_gl_line % the constructed double line (hbox)
\newskip\g_gl_glue % extra glue between glossed pairs or triples
\g_gl_glue = 0pt plus 2pt minus 1pt % allow stretch/shrink between words
%\g_gl_glue = 5pt plus 2pt minus 1pt % allow stretch/shrink between words
\newif\ifnotdone

\def\g_gl_word_one_each{\g_gl_fn_family\itshape} 
\def\g_gl_word_two_each{\g_gl_fn_family\normalfont} 
\def\g_gl_word_three_each{\g_gl_fn_family\normalfont} 
\let\eachwordone=\g_gl_word_one_each
\let\eachwordtwo=\g_gl_word_two_each
\let\eachwordthree=\g_gl_word_three_each

\def\g_gl_word_last#1#2#3% #1 = \each, #2 = line box, #3 = word box
{\setbox#2=\vbox{\unvbox#2%
\global\setbox#3=\lastbox
}%
\ifvoid#3\global\setbox#3=\hbox{#1\strut{} }\fi
% extra space following \strut in case #1 needs a space
}

\def\testdone
{\ifdim\ht\g_gl_line_one=0pt
\ifdim\ht\g_gl_line_two=0pt \notdonefalse % tricky space after pt
\else\notdonetrue
\fi
\else\notdonetrue
\fi
}

\gdef\g_gl_words_get(#1,#2)#3 #4\\% #1=linebox, #2=\each, #3=1st word, #4=remainder
{\setbox#1=\vbox{\hbox{#2\strut#3 }% adds space
\unvbox#1%
}%
\def\more{#4}%
\ifx\more\empty\let\more=\g_gl_words_done
\else\let\more=\g_gl_words_get
\fi
\more(#1,#2)#4\\%
}

\gdef\g_gl_words_done(#1,#2)\\{}%

\gdef\g_gl_twosent#1\\ #2\\{% #1 = first line, #2 = second line
\g_gl_words_get(\g_gl_line_one,\eachwordone)#1 \\%
\g_gl_words_get(\g_gl_line_two,\eachwordtwo)#2 \\%
\loop\g_gl_word_last{\eachwordone}{\g_gl_line_one}{\g_gl_word_one}%
\g_gl_word_last{\eachwordtwo}{\g_gl_line_two}{\g_gl_word_two}%
\global\setbox\g_gl_line=\hbox{\unhbox\g_gl_line
\hskip\g_gl_glue
\vtop{\box\g_gl_word_one   % vtop was vbox
\nointerlineskip
\box\g_gl_word_two
}%
}%
\testdone
\ifnotdone
\repeat
\egroup % matches \bgroup in \gloss
\g_gl_stop\endgroup\printlanguage}

\gdef\g_gl_threesent#1\\ #2\\ #3\\{% #1 = first line, #2 = second line, #3 = third
\g_gl_words_get(\g_gl_line_one,\eachwordone)#1 \\%
\g_gl_words_get(\g_gl_line_two,\eachwordtwo)#2 \\%
\g_gl_words_get(\g_gl_line_three,\eachwordthree)#3 \\%
\loop\g_gl_word_last{\eachwordone}{\g_gl_line_one}{\g_gl_word_one}%
\g_gl_word_last{\eachwordtwo}{\g_gl_line_two}{\g_gl_word_two}%
\g_gl_word_last{\eachwordthree}{\g_gl_line_three}{\g_gl_word_three}%
\global\setbox\g_gl_line=\hbox{\unhbox\g_gl_line
\hskip\g_gl_glue
\vtop{\box\g_gl_word_one   % vtop was vbox
\nointerlineskip
\box\g_gl_word_two
\nointerlineskip
\box\g_gl_word_three
}%
}%
\testdone
\ifnotdone
\repeat
\egroup % matches \bgroup in \gloss
\g_gl_stop\endgroup\printlanguage}

%\def\g_gl_stop{{\hskip -\g_gl_glue}\unhbox\g_gl_line\end{flushleft}}

% \leavevmode puts us back in horizontal mode, so that a \\ will work
\def\g_gl_stop{{\hskip -\g_gl_glue}\unhbox\g_gl_line\leavevmode \egroup}
% End modificatioons

% Defining_New_Commands

% DefiningGeneralCommands
\def\g_gl_fn_family{}
\def\selectglossfont#1{\renewcommand\g_gl_fn_family{#1}}
\def\setmonogloss{
\def\g_gl_word_one_each{\g_gl_fn_family\normalfont} 
\def\g_gl_word_two_each{\g_gl_fn_family\normalfont}
} 
\def\setbigloss{
\def\g_gl_word_one_each{\g_gl_fn_family\itshape} 
\def\g_gl_word_two_each{\g_gl_fn_family\normalfont} 
}
\def\setigloss{
\def\g_gl_word_one_each{\g_gl_fn_family\itshape} 
\def\g_gl_word_two_each{\g_gl_fn_family\normalfont} 
\def\ot{\ctsep\g_gl_lines_single {\printgramjug}}
}

%:> 'printgramjug'
\def\@gramjug{}
\newcommand{\gj}[1]{
\def\@gramjug{#1}
}
\def\printgramjug{\ifthenelse{\equal{\@gramjug}{}}{}{\@gramjug~}}
%<:
% End
% Defining_Low-level_Commands
%Line_Spacing
\def\ctsep{\vspace*{0em}}
\def\ftsep{}
%End
%:> 'ct' : cs for typesetting context of a piece of data
\def\ctname{Context:}% allows customization of the name
\def\ctnamefn#1{#1}% allows customization of the font
\def\ct{\ctnamefn{\ctname}~}
%<:
\newcommand\ot{\def\mb{{\textcolor{white}{\printgramjug}}}\ctsep\g_gl_lines_double {\printgramjug}}%
\newcommand\mb{\ctsep\g_gl_lines_single {\printgramjug}}
\newcommand\gl{\textcolor{white}{\printgramjug}}%
\newcommand\ft{\g_gl_tr \textcolor{white}{\printgramjug}}
\def\lt#1{(Lit.~`#1')}
%:> 'ta' : cs for typesetting response task of a piece of data
\def\taname{Task:}% allows customization of the name
\def\tanamefn#1{#1}% allows customization of the font
\def\ta{\tanamefn{\taname}~}
%<:
%:> 're' : cs for typesetting speaker's response to the response task
\def\rename{Response:}% allows customization of the name
\def\renamefn#1{#1}% allows customization of the font
\def\re{\renamefn{\rename}~}
%<:
\def\id#1{(#1)}
% End

%
% INTERLINEAR 
%

\newenvironment{interlinear}[1][2]%
{%	
\ifthenelse{\equal{#1}{2}}{\setmonogloss}{
\ifthenelse{\equal{#1}{3}}{\setbigloss}{
\ifthenelse{\equal{#1}{i}}{\setigloss}{\setmonogloss\typeout{'lingdiss error' : There are only three options for this argument -- i.e. 2=digloss ; 3=trigloss ; i=digloss with \glosslineone in \itshape}}}}
}{}


%
% DATA (interlinear with built in lex env.)
%

\newenvironment{data}[1][2]
{
\begin{lex}	\ifthenelse{\equal{#1}{2}}{\setmonogloss}{
\ifthenelse{\equal{#1}{3}}{\setbigloss}{
\ifthenelse{\equal{#1}{i}}{\setigloss}{\setmonogloss\typeout{'lingdiss error' : There are only three options for this argument -- i.e. 2=digloss ; 3=trigloss ; i=digloss with \glosslineone in \itshape}}}}
}{\end{lex}}

%
% REF COMMANDS
%

\newcommand{\rnext}[2][1]{\addtocounter{lexno}{#1}(\arabic{lexno}#2)\addtocounter{lexno}{-#1}}
\newcommand{\rlast}[2][1]{\addtocounter{lexno}{-#1}(\arabic{lexno}#2)\addtocounter{lexno}{#1}}

\newcommand{\Next}[1][]{\rnext[1]{#1}}
\newcommand{\NNext}[1][]{\rnext[2]{#1}}
\newcommand{\Last}[1][]{\rlast[0]{#1}}
\newcommand{\LLast}[1][]{\rlast[1]{#1}}

%%%%%%%%%%%%%%%%%%%%%%%%
%  Added formats of examples: %
%%%%%%%%%%%%%%%%%%%%%%%%
%
% \lx {sentence}               (num. ex.)
% \lx[jdg]{sentence}         (num. ex. with judg.)
% 
% \lx {                                 (num. ex.)
%		\ax {sentence}         (num. subex.) 
%		\ax {sentence}         (num. subex.) 
%	} 
% \lx[jdg] {                          (num. ex.)
%		\ax[jdg]{sentence}   (num. subex. with judg.) 
%		\ax[jdg]{sentence}   (num. subex. with judg.) 
%	} 
% \gex {                                 (num. gl. ex.)
%		\ct <context/scenario>\\ % option
%		\mb <sentence>\\
%		\gl <glosses>\\
%		\ft <translation>
%		\lt <litteral translation> % option 
%	} 
% \gex[jdg] {                          (num. gl. ex. with judg.)
%		%\ct <context/scenario>\\ % option
%		\mb <sentence>\\
%		\gl <glosses>\\
%		\ft <translation>
%		\lt <litteral translation> % option 
%	} 
% 
% \lex {                                 (num. ex.)
%		\gix  {                        (num. gl. subex.)
%		\ct <context/scenario>\\ % option
%		\mb <sentence>\\
%		\gl <glosses>\\
%		\ft <translation>
%		\lt <litteral translation> % option 
%	} 
%		\gix  {                          (num. gl. subex.)
%		\ct <context/scenario>\\ % option
%		\mb <sentence>\\
%		\gl <glosses>\\
%		\ft <translation>
%		\lt <litteral translation> % option 
%	} 
%	} 
% \lx {                          (num. ex. with judg.)
%		\gix[jdg] {                    (num. gl. subex. with judg.)
%		\ct <context/scenario>\\ % option
%		\mb <sentence>\\
%		\gl <glosses>\\
%		\ft <translation>
%		\lt <litteral translation> % option 
%		} 
%		\gix[jdg] {                      (num. gl. subex. with judg.)
%		\ct <context/scenario>\\ % option
%		\mb <sentence>\\
%		\gl <glosses>\\
%		\ft <translation>
%		\lt <litteral translation> % option 
%		} 
%} 
% \gex is a variant of \gx
% 
%%%%%%%%%%%%%%%%%%%%%%%

% €€€€€€€€€€€€€€€€€€€€€€€€€
% €€                                             €€
% €€    DEFINING MACROS      €€
% €€                                             €€
% €€€€€€€€€€€€€€€€€€€€€€€€€

 \providecommand{\lx}[2][]{\ex #1 #2  \ex*} 
  \providecommand{\gx}[2][]{
  \begin {data} \gj{#1} #2\end{data}} 
  \let\gex\gx\relax
 
  \newcommand{\ax}[2][]{\ix #1 #2\ix* }
\newcommand{\gix}[2][]{\ix\begin{interlinear}\gj{#1} #2\end{interlinear}\ix*} 


% End
\restoreus
\makeatother 

% todo LINGTHM

\makeatletter%
\@ifpackageloaded{ifthenx}{}{\RequirePackage{ifthenx}}
\newcommand{\defltcounter}[1]{\newcounter{lt#1no}\setcounter{lt#1no}{0}}
%\def\ltlabel{Generalization}
\def\ltlabelfn#1{\textbf{#1}}
\def\ltnostyle#1{\arabic{#1}}
\def\lttitlefn#1{{\normalfont #1}}
\def\lttitlelp{(}
\def\lttitlerp{)}
\def\ltheadsep{\\}


\newcommand{\newlingthm}[2]{
\refstepcounter{lt#1no}
\ltlabelfn{#2~\ltnostyle{lt#1no}}
}

%Définition	ldefin
\defltcounter{defin}
\newenvironment{ldefin}[1][]{\begin{lex} 
\newlingthm{defin}{Definition}
\ifthenelse{\equal{#1}{}}{}{\lttitlelp\lttitlefn{#1}\lttitlerp}
\ltheadsep}
{\end{lex}\relax}

%Conjecture	lconjec
\defltcounter{conjec}
\newenvironment{lconjec}[1][]{\begin{lex} 
\newlingthm{conjec}{Conjecture}
\ifthenelse{\equal{#1}{}}{}{\lttitlelp\lttitlefn{#1}\lttitlerp}
\ltheadsep}
{\end{lex}\relax}

%Test
\defltcounter{diag}
\newenvironment{ldiag}[1][]{\begin{lex} 
\newlingthm{diag}{Test diagnostique}
\ifthenelse{\equal{#1}{}}{}{\lttitlelp\lttitlefn{#1}\lttitlerp}
\ltheadsep}
{\end{lex}\relax}

%Notation
\defltcounter{notat}
\newenvironment{lnotat}[1][]{\begin{lex} 
\newlingthm{notat}{Notation}
\ifthenelse{\equal{#1}{}}{}{\lttitlelp\lttitlefn{#1}\lttitlerp}
\ltheadsep}
{\end{lex}\relax}
%ThéorÚme
\defltcounter{thm}
\newenvironment{lthm}[1][]{\begin{lex} 
\newlingthm{thm}{ThéorÚme}
\ifthenelse{\equal{#1}{}}{}{\lttitlelp\lttitlefn{#1}\lttitlerp}
\ltheadsep}
{\end{lex}\relax}

%Lemme
\defltcounter{lem}
\newenvironment{llem}[1][]{\begin{lex} 
\newlingthm{lem}{Lemme}
\ifthenelse{\equal{#1}{}}{}{\lttitlelp\lttitlefn{#1}\lttitlerp}
\ltheadsep}
{\end{lex}\relax}

%Inférence	linfer
\defltcounter{infer}
\newenvironment{linfer}[1][]{\begin{lex} 
\newlingthm{infer}{Inf\'erence}
\ifthenelse{\equal{#1}{}}{}{\lttitlelp\lttitlefn{#1}\lttitlerp}
\ltheadsep}
{\end{lex}\relax}

% todo traits
\newcommand{\traitanot}[1]{\\ \matricev{#1}}
\def\vmfeature#1{\mbox{\textsc{\MakeLowercase{#1}}}\\}
\newcommand\hmfeature[2][,]{\mbox{\textsc{\MakeLowercase{#2}}}#1~}
\def\tfeature#1{\textsc{\MakeLowercase{#1}}}
\def\matriceh#1{\mbox{{[}\textsc{\MakeLowercase{#1}}{]}}}
\newcommand{\matricev}[1]{
\let\trait\vmfeature\relax
\mbox{%
$
\left[\hspace*{-2.9pt}%
\begin{array}{l}
#1
\end{array}
\hspace*{-2.9pt}\right]
$}
\let\trait\tfeature\relax
}
\newenvironment{fmatrix}
{\let\trait\mfeature\relax
  $\left[\hspace*{-2.9pt}\begin{array}{l}}
{\end{array}\hspace*{-2.9pt}\right]$
 \let\trait\tfeature\relax}

\let\trait\tfeature
\newcommand{\sofeat}[1]{\textsubscript{\textsc{\MakeUppercase{#1}}}}

% todo semantix
